\documentclass[blue]{beamer}
%\documentclass[handout]{beamer}

%\beamerboxesdeclarecolorscheme{alert}{brown}{brown!25!averagebackgroundcolor}

\usepackage{beamerthemeshadow}
\usepackage[ngerman]{babel}
\usepackage[ansinew]{inputenc}
\usepackage{html,makeidx}

\hypersetup{colorlinks=true,pagecolor=blue,urlcolor=blue} % linkcolor=green,filecolor=red

%%%%%%%%%%%%%%%%%%%%%%%%%%%%%%%%%%%%%%%%%%%%%%%%%%%%%%%%%%%%%%%%%%%%%%%%%%%%%%%%%%%%
%%%%%%%%%%%%%%%%%%%%%%%%%%%%%%%%%%%%%%%%%%%%%%%%%%%%%%%%%%%%%%%%%%%%%%%%%%%%%%%%%%%%

\title{METANET - Interactive Knowledge Network}
\institute[META-D.O.N]{META-D.O.N\\Association for Cultural Substitution Services}
\author{Hannes Weingartner}
\date{April 17, 2010}

\begin{document}
\frame{\titlepage}

% no entry in TOC but in the navigation bar
%\section*{overview}
\setcounter{tocdepth}{1}
\frame{\tableofcontents}


%%%%%%%%%%%%%%%%%%%%%%%%%%%%%%%%%%%%%%%%%%%%%%%%%%%%%%%%%%%%%%%%%%%%%%%%%%%%%%%%%%%%
%%%%%%%%%%%%%%%%%%%%%%%%%%%%%%%%%%%%%%%%%%%%%%%%%%%%%%%%%%%%%%%%%%%%%%%%%%%%%%%%%%%%


\section{project}
\subsection{introduction}
\frame
{
\frametitle{\textbf{project introduction}}
\begin{itemize}
\item the technical aspect of a project of the \textbf{Meta-D.O.N Association}: sociocultural studies in serbia called \textbf{New[B]Order} - \htmladdnormallink{http://www.meta-don.org/newborder/}{http://www.meta-don.org/newborder/}
\item project on the topography of the centre-periphery-complex in post-communist Eastern Europe
\item metastatic penetration of social spaces scientifically measured to create quantitative and qualitative indicators for social disintegration
\end{itemize}
}

\frame
{
\frametitle{\textbf{project introduction}}
\begin{itemize}
	\item e.g. indicators
	\begin{itemize}
		\item privatization aspects of companies in serbia - \textbf{Freedom Fighters Collective} (\htmladdnormallink{http://www.freedomfight.net/cms/}{http://www.freedomfight.net/cms/})
		\item subversive signs of the periphery
		\item turbo folk politics
	\end{itemize}
	\item capture indicators cartographically with a link to a visual semantic database
	\item concept of an interactive geo-spatial \textbf{knowledge network}
	\item web application based on \textbf{free- and open source software components}
\end{itemize}
}


\subsection{goal}
\frame
{
\frametitle{\textbf{project goal}}
\begin{itemize}
\item \textbf{flexiblity in usage:} generic tool for scientific research projects
\item \textbf{administration:} intuitive CMS interface for researchers
\item TODO
\item algorithms for finding similarities/references between artifacts
\item \textbf{usability:} advanced hardware and software interfaces to the user (e.g for exhibitions)
\end{itemize}
}


\subsection{domains}
\frame
{
\frametitle{\textbf{project domains}}
\begin{itemize}
\item \textbf{metablog - artifacts input:}
  \begin{itemize}
    \item text, sound, images, an mm data through CMS
    \item mobile devices for geo-tagged content input
  \end{itemize}
\item \textbf{metamap:} 2D map for customized geo-spatial representation of artifacts
\item \textbf{metaspace:} 2D/3D semantic graph/subgraph associated with map artifacts
\end{itemize}
}

\frame
{
\frametitle{\textbf{project domains}}
\includegraphics[width=0.85\textwidth]{bin/domains/domains.png}
}


\subsection{challenges}
\frame
{
\frametitle{\textbf{project challenges}}
\begin{itemize}
\item \textbf{customization of map} for adequate content representation
\item \textbf{administration:} intuitive CMS interface for researchers
\item \textbf{data consistency} between metamap and metaspace
\item \textbf{metaspace hardware HCI} design and implementation
\end{itemize}
}


\subsection{technologies}
\frame
{
\frametitle{\textbf{project technologies}}
\begin{itemize}
\item \textbf{language:} java and java script
\item \textbf{dependencies:}
  \begin{itemize}
    \item \textbf{Semaspace} (\htmladdnormallink{http://residence.aec.at/didi/FLweb/}{http://residence.aec.at/didi/FLweb/})
    \item \textbf{OpenStreetMap} aka OSM (\htmladdnormallink{http://www.openstreetmap.org/}{http://www.openstreetmap.org/})
    \item \textbf{Google Web Toolkit 2.0} aka (\htmladdnormallink{http://code.google.com/webtoolkit/}{http://code.google.com/webtoolkit/})
  \end{itemize}
\end{itemize}
}

%%%%%%%%%%%%%%%%%%%%%%%%%%%%%%%%%%%%%%%%%%%%%%%%%%%%%%%%%%%%%%%%%%%%%%%%%%%%%%%%%%%%
%%%%%%%%%%%%%%%%%%%%%%%%%%%%%%%%%%%%%%%%%%%%%%%%%%%%%%%%%%%%%%%%%%%%%%%%%%%%%%%%%%%%

% TODO research:  knowledge network
\section{knowledge networks}
\subsection{theory}
\frame
{
\frametitle{\textbf{knowledge networks theory}}
\includegraphics[width=0.85\textwidth]{bin/semaspace/semaspace.png}
}

\frame
{
\frametitle{\textbf{knowledge networks theory}}
\begin{itemize}
	\item \textbf{TODO} 
\end{itemize}
}




%%%%%%%%%%%%%%%%%%%%%%%%%%%%%%%%%%%%%%%%%%%%%%%%%%%%%%%%%%%%%%%%%%%%%%%%%%%%%%%%%%%%
%%%%%%%%%%%%%%%%%%%%%%%%%%%%%%%%%%%%%%%%%%%%%%%%%%%%%%%%%%%%%%%%%%%%%%%%%%%%%%%%%%%%

% TODO research: virtools technical + developers
\section{semaspace}
\subsection{outline}
\frame
{
\frametitle{\textbf{semaspace outline}}
\begin{itemize}
	\item \textbf{fast opengl accelerated graph editor and browser} for large knowledge networks with more than 10000 text-fields
	\item developed by \textbf{Dietmar Offenhuber and Gerhard Dirmoser - FH Hagenberg}
	\item \textbf{interactive graph layout} in 2d and 3d
	\item semaspace fills the gap between complex tools for network analysis and simple graph editors or mind-mapping tools (Dirmoser).
	\item Gerhard Dirmoser creates complex large-scale wall diagrams on topics like \textbf{25 years of ars electronica or performance art} (2003)
\end{itemize}
}

\subsection{outline}
\frame
{
\frametitle{\textbf{semaspace outline}}
\begin{itemize}
	\item manually \textbf{define the nodes overall shape} of the network by placing and subsequently locking the position of individual nodes
	\item \textbf{nodes} can incorporate data such as images, sound and text (added while navigating the graph)
	\item ui for \textbf{web clients} and \textbf{exhibitions}
	\item \textbf{java swt} desktop application
	\item embedding into web clients with \textbf{virtools webplayer}
\end{itemize}
}

\subsection{dependencies}
\frame
{
\frametitle{\textbf{semaspace dependencies}}
\begin{itemize}
	\item \textbf{jogl:} open source java binding to opengl api from sun (for harware supported 3d graphics)
	\item \textbf{jftgl:} java based lib for accessing tt fonts within opengl
	\item \textbf{apache batik svg toolkit:} open source java based toolkit for using images in the svg format (xml based)
	\item \textbf{\htmladdnormallink{semaspace demo}{file:///C:/Users/hn/Workspace-Java/metanet/files/presentation/bin/semaspace/web-demo/semaspace.html}}
\end{itemize}
}


\subsection{extensions}
\frame
{
\frametitle{\textbf{semaspace extensions}}
\begin{itemize}
	\item \textbf{video nodes}
	\item \textbf{remote graph access and manipulation} through network connections
	\item research and implementation of \textbf{extending hardware interfaces} for browsing
		\begin{itemize}[<+-|alert@+>]
			\item browsing (zoom, rotation)
			\item 2D/3D
			\item fog
			\item freeze
			\item node distance
		\end{itemize}
\end{itemize}
}


%%%%%%%%%%%%%%%%%%%%%%%%%%%%%%%%%%%%%%%%%%%%%%%%%%%%%%%%%%%%%%%%%%%%%%%%%%%%%%%%%%%%
%%%%%%%%%%%%%%%%%%%%%%%%%%%%%%%%%%%%%%%%%%%%%%%%%%%%%%%%%%%%%%%%%%%%%%%%%%%%%%%%%%%%

\section{openstreetmap}
\subsection{outline}
\frame
{
\frametitle{\textbf{OSM outline}}
\begin{itemize}
	\item \textbf{OSM: open source map solution} for web clients
	\item everyone can extend the map
	\item collecting GPS coordinates
	\item like wikipedia for map
	\item \textbf{no legal or technical restrictions} on their use - full creativity
	\item other implementations do not allow highly customized overlays
\end{itemize}
}

\subsection{strategy}
\frame
{
\frametitle{\textbf{OSM creating data}}
\begin{itemize}
	\item \textbf{creating OSM data} with OSM map editors:
	\begin{itemize}
			\item java OSM (JOSM) - desktop app
			\item potlatch - flash
			\item merkaartor - for UNIX, WIN, MAC
	\end{itemize}
	\item \textbf{adding tags} (node, linear, area) for OSM data do be rendered and upload
	\item \textbf{slippy map:} OSM default web interface for browsing rendered OSM data
\end{itemize}
}

\frame
{
\frametitle{\textbf{OSM rendering}}
\begin{itemize}
\item \textbf{generating rendered graphics} on own computer by using tools:
	\begin{itemize}
			\item \textbf{kosmos:} lightweight OSM map rendering platform for WIN
			\item \textbf{osmarender:} rendering platform for generating SVG image for OSM data based on XSLT
			\item \textbf{mapnik:} opensource toolkit in c++ and pyhton, all OS platforms
			\begin{itemize}
				\item default for OSM - mapnik renders 256 x 256 px tiles served from the OSM tile server
				\item XML file holds rendering props like color, withs of lines, etc.
				\item rendered tiles are referenced by URLs through JS from web client
				\item raw OSM data can be downloaded for setting up an own tile server (e.g. planet.osm)
				\item uses PostgreSQL / PostGIS as spatial database (osm2pgsql converter)
		\end{itemize}
	\end{itemize}
\end{itemize}
}

\subsection{embedding}
\frame
{
\frametitle{\textbf{OSM embedding}}
\begin{itemize}
	\item \textbf{integration in web page} like Google, Yahoo maps by using JS libs for customize the map and interface
	\begin{itemize}
		\item \textbf{OpenLayers:} feature-rich, open source, most popular for OSM maps, BSD lic. (\htmladdnormallink{www.openlayers.org}{www.openlayers.org})
		\begin{itemize}
			\item dynamic map for the web page
			\item displays tiles and markers from any source
			\item separation between tools and data
			\item project of Open Source Geospatial Foundation (OSGeo)
		\end{itemize}
		\item \textbf{Mapstraction:} JS lib wrapper for multiple implementations (e.g. OL, Google, etc.)
	\end{itemize}
	\item several project on google code with topic OSM on android
	\item integration of OL in GWT through open source project \textbf{GWT-OpenLayers} (\htmladdnormallink{http://sourceforge.net/projects/gwt-openlayers/}{http://sourceforge.net/projects/gwt-openlayers/})
\end{itemize}
}


%%%%%%%%%%%%%%%%%%%%%%%%%%%%%%%%%%%%%%%%%%%%%%%%%%%%%%%%%%%%%%%%%%%%%%%%%%%%%%%%%%%%
%%%%%%%%%%%%%%%%%%%%%%%%%%%%%%%%%%%%%%%%%%%%%%%%%%%%%%%%%%%%%%%%%%%%%%%%%%%%%%%%%%%%


\section{realization}
\subsection{architecture}
\frame
{
\frametitle{\textbf{realization architecture}}
\includegraphics[width=0.85\textwidth]{bin/architecture/component-diagram_update.png}
}

\subsection{interface}
\frame
{
\frametitle{\textbf{realization interface}}
\includegraphics[width=0.85\textwidth]{bin/features/metamap/01_gui_metamap_start_update.png}
}

\frame
{
\frametitle{\textbf{realization interface}}
\includegraphics[width=0.85\textwidth]{bin/features/metamap/02_gui_metamap_admin_start_update.png}
}

%%%%%%%%%%%%%%%%%%%%%%%%%%%%%%%%%%%%%%%%%%%%%%%%%%%%%%%%%%%%%%%%%%%%%%%%%%%%%%%%%%%%
%%%%%%%%%%%%%%%%%%%%%%%%%%%%%%%%%%%%%%%%%%%%%%%%%%%%%%%%%%%%%%%%%%%%%%%%%%%%%%%%%%%%

%\section*{Summary}
\frame
{
\frametitle{\textbf{Summary}}
\begin{itemize}
\item \textbf{METANET - Interactive Knowledge Network}
\end{itemize}
}

%%%%%%%%%%%%%%%%%%%%%%%%%%%%%%%%%%%%%%%%%%%%%%%%%%%%%%%%%%%%%%%%%%%%%%%%%%%%%%%%%%%%

\frame[plain]
{
\begin{center}
{\Large\textbf{Thank You.}}
\end{center}
}

\end{document}

%%%%%%%%%%%%%%%%%%%%%%%%%%%%%%%%%%%%%%%%%%%%%%%%%%%%%%%%%%%%%%%%%%%%%%%%%%%%%%%%%%%%


%\begin{beamerboxesrounded}[scheme=alert,shadow=true]{simple.xul}
%\end{beamerboxesrounded}
%\hyperlink{packages<2>}{\beamergotobutton{packages item 2}}


%\frame
%{
%\frametitle{\textbf{title}}
%\begin{itemize}
%\item<1-|alert@1>\textbf{item}
%	\begin{itemize}
%		\item<3-|alert@3> item
%	\end{itemize}
%\item<2-|alert@2>\textbf{item}
%	\begin{itemize}
%		\item<4-|alert@4> item
%	\end{itemize}
%\end{itemize}
%}




